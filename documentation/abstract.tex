
Pasaulinis tinklas (WEB) tapo viena iš didžiausių internetinių puslapių šaltinių saugykla. Puslapių turinys gali būti klasifikuojamas į kategorijas - tokiu būdu internetiniams puslapiams yra sukuriama struktūra ir juos lengviau indeksuoti, dėl to žmonėms surasti informaciją pasauliniame tinkle yra lengviau. Šiame straipsnyje, internetinių puslapių klasifikavimas yra atliekamas naudojantis save mokinančios kompiuterinės sistemos ir natūralios kalbos apdorojimo metodais, kuriais naudojantis yra sukuriami modeliai su internetinių puslapių klasifikavimo funkcionalumu, naudojantis internetinių puslapių turiniu. Internetinių puslapių klasifikavimo modeliai yra sukurti remiantis populiauriausių žodžių aibėmis kiekvienai internetinio puslapio kategorijai ir kiekvieno internetinio puslapio turinio žodžių aibėmis, kurios yra transformuojamos į save mokinančios kompiuterinės sistemos savybių aibę, naudojantis natūralios kalbos apdorojimo metodais. Modelių sugebėjimai klasifkuojant internetines svetaines buvo išbandomi naudojantis pasaulinio tinklo (WEB) anglų kalbos turinio internetinėmis svetainėmis. Geriausias sukurtas modelis geba klasifikuoti internetinius puslapius $\sim$ 70 \% tikslumu.

\bigskip
\textbf{Raktiniai žodžiai}: \textit{Save mokinačios kompiuterinės sistemos, Natūralios kalbos apdorojimas, Automatinis informacijos surinkimas, WEB klasifikavimas, Duomenų apdorojimas, Statistika}
