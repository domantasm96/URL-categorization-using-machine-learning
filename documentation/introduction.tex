

Website classification is a well known problem in computer science field. World Wide Web (WEB) consist of a lot of websites with a different categories. The motivation of classified websites would be significant - that would allow index websites categories by their content so it would be easier for people to find information among many websites.


The article is oriented to the two big topics in the data science: Machine Learning and Natural Language Processing. Article consist of three main parts - first two topics is theory based and the third topic is an implementation part:

\begin{enumerate}
    \item \textbf{Natural Language Processing (NLP)}
    
    Natural Language Processing topic is oriented to introduce fundamentals of natural language processing methods. In this section is explained NLP types, problems and methods that are used in the implementation part.
    
    \item \textbf{Machine Learning (ML)}
    
    Machine Learning topic is oriented to introduce fundamentals of machine learning theory. Since machine learning by itself is a big field of research, the structure of this section is divided into three parts:
    \begin{enumerate}
        \item Supervised Learning.
        \item Unsupervised Learning.
        \item Machine Learning models evaluation.
    \end{enumerate}
    
    \item \textbf{Machine Learning and Natural Language Processing methods implementation in practise}
    
    In implementation part there is explained of how data sets are preprocessed using NLP methods and how models are trained to be capable of classifying real world websites. Methods used in the implementation part by creating models are covered in theories part so all sections are related to each other.
\end{enumerate}

The main goal of this article is to explain all process for creating models which are capable of predicting websites categories based on websites content features.


Results of this project: created machine learning and custom models that are capable of classifying english content websites. Models performances are tested on a different websites data sets which two main factors of each data set is Website URL with website category. Machine Learning models performance evaluation on original data set was used Accuracy, Precision, Recall, F1 scores while on custom human made data set was, models performance was calculated by correctly and incorrectly predicted categories websites ratio.

